\input{pre/themainpreumble}
	\def\authors{Есюнин М.В., Есюнин Д.В.}
	\def\labname{Исследование отражательного клистрона}
	\def\sciadviser{Павличенко И.А.}
	\def\shortlabname{Исследование отражательного клистрона}
\usepackage{float}
\usepackage{physics}
\begin{document}
\input{pre/titlepage}
% \renewcommand{\vec}{\mathbf}

\renewcommand{\phi}{\varphi}
\renewcommand{\hat}{\widehat}

\tableofcontents

\newpage
\sloppy

\section{Теоретическая часть}
\subsection{Введение}
Отражательный клистрон
\footnote{Название <<клистрон>> происходит от греческого слова «клизо», что означает береговой волнорез.}, 
изучению которого посвящена лабораторная работа, предназначен для генерации электромагнитных колебаний СВЧ диапазона. Исторически именно генераторы клистронного типа позволили освоить диапазоны сантиметровых и дециметровых волн, в которых традицион­ные электронные лампы оказались неэффективными.

Генерация в клистронах осуществляется за счет преобразования кинетической энергии пучка электронов, ускоренных статическим электрическим полем, в энергию сверхвысокочастотных колебаний. Для получения СВЧ колебаний пучок электронов модулируется по плотности и проводится че­ рез резонатор клистрона, в котором переменный конвекционный ток пучка возбуждает колебания.

Неоднородный электронный поток в приборах клистронного типа создается посредством так называемого динамического способа управления плотностью зарядов. При этом способе управления однородный электронный по­ток пропускается через переменное электрическое поле управляющего (группирующего) устройства. Переменное поле воздействует на скорости электро­нов в потоке, периодически ускоряя и замедляя их движение в зависимости от фазы высокочастотного поля, существующей в момент пролета электронов через управляющее устройство. Возникающее различие в скоростях электронов приводит к их группировке при последующем движении за пре­делами управляющего устройства. При этом в определенной точке пространства в определенный момент времени происходит образование электронного сгустка. Поток приобретает пульсирующий характер, причем электронные уплотнения и разрежения в данной точке пространства возникают с периодичностью, соответствующей частоте управляющего поля
\footnote{При использовании приемов т. н. электростатического управления основной результат управления состоит в модуляции электронного потока по плотности непосредственно в области управляющего поля.}.

Характер процесса группировки электронов, координата и время образования сгустка в клистронах определяются условиями движения вне управляющего промежутка.

В пролетных клистронах группировка происходит при движении электронов по инерции в пространстве, свободном от внешних постоянных или переменных полей; группировка становится возможной за счет того, что «быстрые» электроны догоняют «медленные», вылетевшие из управляюще­го промежутка раньше «быстрых».

В отражательных клистронах электроны, вышедшие из группирующего устройства, движутся в постоянном тормозящем электрическом поле. Значение этого поля таково, что все электроны, вылетевшие из управляющего устройства, возвращаются назад. Группировка в отражательных клистронах происходит благодаря тому, что «быстрые» электроны находятся в пространстве группировки дольше «медленных», и на обратном пути оказывается возможной встреча «быстрых» электронов с медленными, вышедшими из управляющего промежутка позже «быстрых».

Принципиальная схема отражательного клистрона представлена на рис. \ref{fig:1}. Функции управляющего устройства и устройства, накапливающего энергию электромагнитных колебаний в отражательном клистроне, объединены здесь в тороидальном резонаторе.

\begin{figure}[h!]
	\centering
	\includegraphics[width=0.4\textwidth]{text/fig1}
	\caption{Идеализированная принципиальная схема клистрона: 1 — катод, 2 — резонатор, 3 — отражатель, 4 — вывод энергии, 5 — электронный поток, 6 — управляющий электрод}
	\label{fig:1}
\end{figure}

Электронный поток, эмитируемый катодом, ускоряется в промежутке между катодом и резонатором, после чего первый раз попадает в зазор резонатора, образованный двумя прозрачными для потока металлическими сетками. В зазоре происходит модуляция электронного потока по скорости. После выхода из резонатора электроны движутся в тормозящем поле отражающего электрода, возвращаются назад и повторно проходят зазор. При соответствующем выборе потенциалов на электродах клистрона сгусток электронов формируется в сечении зазора резонатора в момент времени, когда высокочастотное поле в зазоре тормозит возвращающиеся электроны, при этом происходит преобразование кинетической энергии электронов в энергию колебаний резонатора. Существование стационарных колебаний в клистроне оказывается возможным при компенсации потерь и резонаторе и нагрузке притоком энергии от модулированного электронного потока.

\subsection{Резонатор клистрона}

Резонатор в клистроне выполняет две функции: служит для модуляции электронного потока по скорости и для преобразования кинетической энергии пучка электронов в энергию электромагнитного ноля. Использовать для выполнении этих функций резонатор, геометрические размеры которого сравнимы с длиной волны основного колебания, возбуждаемого в них, невозможно из-за низкой эффективности взаимодействия электромагнитного поля и пучка электронов, пролетающих через этот резонатор
\footnote{Следует отметить, что существует способ группировки электронного потока по плотности, при котором модулируемый поток проводится через область, в которой фаза управляющего поля успевает многократно измениться за время пролета электрона. При использовании этого способа скорости электронов на выходе из управляющего поля окатываются практически одинаковыми, а электронный поток промодулированным по плотности.}. Действительно. если средняя скорость электронов $v$ значительно меньше скорости света $с$, то время их пролета $t = a/v$ через резонатор, имеющий форму ку­ ба со стороной а, намного превосходит период колебания поля $T =\sqrt{2}a/c$. Электрон при движении через такой резонатор то ускоряется, то тормозится, поэтому обмен энергией между ним и полем незначителен. Время пролета можно уменьшить, взяв вместо куба призму с достаточно малой высотой:
при неизменной низшей частоте колебаний время пролета электрона через резонатор уменьшается. Однако сокращение объема резонатора приводит к
уменьшению его добротности. Использование в генераторе низкодобротной колебательной системы не позволяет производить эффективное преобразование кинетической энергии электронов в энергию электромагнитных колебаний и не обеспечивает стабильности частоты генератора.

Для увеличения объема резонатора и его добротности необходимо, оставляя длину пролетного промежутка малой, создавать дополнительные резервуары энергии. Именно такой резонатор используется в отражательном клистроне.

Схема резонатора представлена на рис. \ref{fig:2}. Геометрические размеры резонатора удовлетворяют неравенствам $a,b \ll \lambda /4,l \ll \lambda /4,d \ll l$, где $\lambda$ — длина волны, соответствующая частоте основной моды резонатора. По­скольку размеры резонатора много меньше длины волны на частоте основ­
ного колебания, резонатор можно рассматривать как колебательный контур с сосредоточенными параметрами.

При этом в резонаторе можно выделить емкостной объем, заключенный между сетками резонатора, и индуктивный объем, т. е. собственно тороидальную часть.

\begin{figure}[H]
	\centering
	\includegraphics[width=0.5\textwidth]{text/fig2}
	\caption{Поперечное сечение тороидального резонатора: 1 — емкостной объем резонатора, 2 — индуктивный объем резонатора}
	\label{fig:2}
\end{figure}

При наличии потерь энергии поля в резонаторе, возникающих из-за конечной проводимости стенок и отвода мощности в нагрузку, свободные колебания резонатора являются затухающими, при этом в рамках метода комплексных амплитуд удобно вводить комплексную частоту $\omega = w^{\prime}+i\omega ^{\prime \prime}$. Ве­
личина $\omega ^{ \prime \prime }$ может быть выражена через добротность резонатора, которая, как правило, измеряется экспериментально.

В клистронах применяются резонаторы, допускающие механическую перестройку резонансной частоты. Наиболее распространены два способа перестройки: индуктивный и емкостной. Индуктивная перестройка осуществляется введением в область магнитного поля металлических поршней, кото­рые, уменьшая занятый полем объем, уменьшают индуктивность резонатора и тем самым повышают его резонансную частоту. Емкостная перестройка может быть осуществлена путем деформации гибкой мембраны. Сближение сеток резонатора, обеспечиваемое прогибом мембраны, ведет к увеличению емкости зазора и уменьшению резонансной частоты. Механические способы перестройки обеспечивают отстройку от основной частоты на 20-25\%.

\subsection{Модуляция скорости электронов в пучке}

Рассмотрим процесс модуляции электронного потока по скорости. В установившемся режиме колебания в резонаторе близки к гармоническим, поле в зазоре имеет вид 
$E = E _ { z } = E _ { m } \text { sin } \omega t$, 
где $E _ { m } = \frac { U _ { m } } { d }$, $U _ { m }$ амплитуда напряжения на зазоре,$E _ { z }$ проекция электрического ноля на ось $z$,
$d$ - ширина зазора, $\omega$ частота стационарных колебаний. Найдем приращение энергии электрона, прошедшего через зазор. Кинетическая энергия,
приобретаемая одиночным электроном при прохождении пути $\dd z$ внутри за­зора, равна работе силы электрического поля: 
$\dd W = - e \frac { U _ { m } } { d } \sin \omega t \dd z$, 
где $e$ - модуль заряда электрона (знак заряда учтен в явном виде)
\footnote{Если напряжение (потенциал второй сетки относительно первой) $U = -U_m \sin \omega t$ положительно, оно ускоряет электрон, движущийся в положительном направлении оси z.}. 
Электрон, прошедший расстояние между катодом и резонатором, имеет кинетическую энергию 
$\frac { m v _ { 0 } ^ { 2 } } { 2 } = e U _ { 0 }$ , где $U_0 > 0$-- потенциал резонатора относительно ка­тода, $v_0$ — скорость электрона, $m$ — его масса. Отсюда при условии $v _ { 0 } \ll c$ имеем
\footnote{Пренебрежение релятивистскими поправками возможно до значений $U_0$ порядка нескольких
десятков киловольт.} 
$v _ { 0 } = \sqrt { \frac { 2 e U _ { 0 } } { m } }$

При выполнении условия 
$\frac { U _ { m } } { U _ { 0 } } \ll 1$ возмущения скорости $v_0$ под воздействием высокочастотного поля незначительны. Поэтому если $t_0$ — время про­
хождения электроном центра зазора, то время его нахождения в точке с координатой $z$ есть $t = t _ { 0 } + \frac { z } { v _ { 0 } }$. Полное приращение энергии имеет вид

\begin{equation}
	 { \Delta W = \int\limits _ { - d / 2 }^{{ + d / 2 }} \frac { - e U _ { m } } { d } \sin \left( \omega t _ { 0 } + \frac { \omega z } { v _ { 0 } } \right) d z = } \\ 
	 { - e U _ { m } \sin \omega t _ { 0 } \frac { \sin \left( \theta _ { 3 } / 2 \right) } { \theta_ { 3 } / 2 } = - e M U _ { m } \sin \omega t _ { 0 }, } 
	\label{eq:1}
\end{equation}

где $\theta _ { 3 } = \frac { \omega d } { t _ { 0 } }$ - невозмущенный угол пролета электрона через модулирующий зазор, $M = \frac { \sin \left( \theta _ { 3 } / 2 \right) } { \theta _ { 3 } / 2 }$ - коэффициент взаимодействия электронного потока с полем зазора. Полная кинетическая энергия электрона, вошедшего в зазор с начальной скоростью на выходе из него может быть представлена в виде:
\begin{equation}
	W = \frac { m v ^ { 2 } } { 2 } = \frac { m v_0 ^ { 2 } } { 2 } + \Delta W = e U _ { 0 } + \Delta W.
	\label{eq:2}
\end{equation}
Таким образом, скорость электрона на выходе из зазора оказывается равной
\begin{equation}
	v = \sqrt { \frac { 2 e } { m } U _ { 0 } } \left( 1 - \frac { M U _ { m } } { U _ { 0 } } \sin \omega t _ { 0 } \right) ^ { 1 / 2 }.
	\label{eq:3}
\end{equation}
При выполнении условия $\frac { U _ { m } } { U _ { 0 } } \ll 1$ можно разложить выражение для $v$ в ряд Тейлора и ограничиться двумя первыми членами ряда:
\begin{equation}
	v = \sqrt { \frac { 2 e } { m } U _ { 0 } } \left( 1 - \frac { 1 } { 2 } \frac { M U _ { m } } { U _ { 0 } } \sin \omega t _ { 0 } + \ldots \right) \approx v _ { 0 } - v _ { 1 } \sin \omega t _ { 0 }
	\label{eq:4}
\end{equation} где $v _ { 1 } = \frac { M U _ { m } } { 2 U _ { 0 } } v _ { 0 }$
Из полученного выражения видно, что скорость электро­на на выходе из зазора определяется фазой поля, существовавшей в момент прохождения им центра зазора. Наибольшая амплитуда скоростной модуляции $v_1$ достигается при стремлении коэффициента взаимодействия $M$ к единице, что выполняется при стремлении $\theta_3$ к нулю.

\subsection{Модуляция электронного пучка по плотности}

В пространстве между резонатором и отражателем электроны двигаются в статическом тормозящем поле. Уравнение движения имеет вид 
$mz''- e \frac { U _ { 0 } - U _ { \text{отр} } } { L }$ где $U_{\text{отр}} < 0$ — напряжение на отражателе, $L$ — расстояние между резонатором и отражателем. Интегрируя первый раз уравнение движения и учитывая выражение для скорости $v$ при выходе из зазора, имеем
\begin{equation}
	z ^ { \prime } = v - \frac { e } { m } \frac { U _ { 0 } - U _ { \text { отр } } } { L } \left( t - t ^ { \prime } \right)
	\label{eq:5}
\end{equation}
где $t'$ — момент выхода электрона из зазора. Через t в дальнейшем будем обозначать момент, когда тот же электрон возвращается в плоскость второй сетки. Повторное интегрирование уравнения \eqref{eq:5} дает
\begin{equation}
	z = v \left( t - t ^ { \prime } \right) - \frac { e } { m } \frac { U _ { 0 } - U _ { \text { отр } } } { L } \frac { \left( t - t ^ { \prime } \right) ^ { 2 } } { 2 } + \frac { d } { 2 }
	\label{eq:6}
\end{equation}

Время пролета электрона в пространстве группировки можно найти из условия $z = d / 2$ при $t=t''$, откуда следует, что 
$t ^ { \prime \prime } - t ^ { \prime } = 0$ и 
$\frac { e } { m } \frac { U _ { 0 } - U _ { \text { отр } } } { L }\cdot \frac { \left( t ^ { \prime \prime } - t ^ { \prime } \right) } { 2 v } = 1$
Первое решение $(t' = t'' )$ соответствует моменту вылета электрона из зазора, второе дает время его пролета в тормозящем поле:
\begin{equation}
	t ^ { \prime \prime } - t ^ { \prime } = \frac { 2 m } { e } \frac { v L } { L _ { 0 } - U _ { \text { отр } } }
	\label{eq:7}
\end{equation}

При выполнении условия $U_{ m } / U _{ 0 } \ll 1$ время пролета в зазоре определяется скоростью $v_0$, поэтому связь времени вылета электрона из зазора со временем его нахождения в центре зазора можно приближенно записать в виде
$t'' = t _ { 0 } + d / \left( 2 v _ { 0 } \right)$. Таким образом, подставляя выражение для скорости \eqref{eq:4} в \eqref{eq:7}, получим
\begin{equation}
	t ^ { \prime \prime } = t ^ { \prime } + \frac { 2 m L } { e \left( U _ { 0 } - U _ { \text { отр } } \right) } \left( v _ { 0 } - \frac { M U _ { m } } { 2 U _ { 0 } } v _ { 0 } \sin \left( \omega t ^ { \prime } - \frac { \omega d } { 2 v _ { 0 } } \right) \right)
	\label{eq:8}
\end{equation}

Из найденного выражения видно, что время пролета электронов в пространстве группировки зависит от фазы высокочастотного напряжения, существующего в момент пролета электроном середины зазора, и от амплитуды этого напряжения.

На рис.\ref{fig:3} представлены примеры пространственно-временных диаграмм(зависимостей координат электронов от времени) для таких значений потенциалов на резонаторе и отражателе, при которых электроны, вышедшие из резонатора в различные моменты времени, возвращаются в него одновремен­но, образуя сгусток. Сверхвысокочастотное поле в резонаторе в рассматриваемых в случаях в момент времени пролета сгустка максимально и является для него тормозящим. Уменьшение кинетической энергии электронов при этом приводит к возрастанию энергии СВЧ поля. Разумеется, при других значениях потенциалов на электродах клистрона сгусток может сформироваться и вне зазора резонатора, при этом эффективная передача энергии от пучка полю становится невозможной.

Домножим соотношение \eqref{eq:8} на $\omega$ и введем следующие величины:
\begin{equation}
	\theta _ { \text{г} } = \frac { 2 m } { e } \frac { v _ { 0 } \omega L } { U _ { 0 } - U _ { \text{ oтр } } }
	\label{eq:9}
\end{equation} — угол пролета электрона в пространстве группировки,
\begin{equation}
	X = \theta _ { \text{г} } \frac{M U _ { m } } { 2 U _ { 0 } }
	\label{eq:10}
\end{equation} — параметр группировки. В результате получим выражение, связывающее время возвращения электрона в плоскость второй сетки $(t'' )$ со временем выхода электрона из зазора $(t')$:
\begin{equation}
	\omega t ^ { \prime \prime } = \omega t ^ { \prime } + \theta _ { r } - X \sin \left( \omega t ^ { \prime } - \frac { \theta _ { 3 } } { 2 } \right)
	\label{eq:11}
\end{equation}
Соотношение \eqref{eq:11} является исходным для нахождения конвекционного тока пучка электронов.

\begin{figure}[h!]
	\centering
	\includegraphics[width=0.5\textwidth]{text/fig3}
	\caption{Пространственно-временные диаграммы движения электронов при двух значениях оптимального времени пролета $\tau$ в пространстве группировки ($z=0$ координата, соответствующая середине зазора)}
	\label{fig:3}
\end{figure}

\subsection{Возбуждение резонатора клистрона током пучка}
Сгруппированный электронный пучок формирует ток $I$, возбуждающий колебания в резонаторе. При этом условие возбуждения клистрона принимает следующий вид:
\begin{equation}
	\pi + 2 \pi n < \left( \theta _ { 3 } + \theta _ { \text{г} } \right) < 2 \pi ( n + 1 )
	\label{eq:12}
\end{equation}где $n=1,2,3,\dots $

Указанные интервалы углов пролета, при которых возможна генерация, носят название зон генерации клистрона. В центре зон генерации, т. е. при
$\theta _ { \text{г} } + \theta _ { \text{з} } = 2 \pi ( n + 3 / 4 )$ амплитуда максимальна, а на краях равна нулю. Угол пролета в пространстве группировки $\theta _ { \text{г} }$ можно варьировать, изменяя ли­бо напряжение на резонаторе, либо напряжение на отражателе (см. \eqref{eq:9}).
Пространственно-временные диаграммы, соответствующие двум оптималь­ным углам пролета, представлены на рис.\ref{fig:3}.

Одной из важных характеристик клистрона является пусковой ток — значение тока $I_0$, при котором клистрон возбуждается:

\begin{equation}
	I _ { 0 \, \text { пуск } } = - \frac { 2 U _ { 0 } G } { M ^ { 2 } 
	\theta _ { \text{ г } } \sin \left( \theta _ { \text{з} } + \theta _ { \text{ г } } \right) }
	\label{eq:13}
\end{equation} где G — действительная часть проводимости резонатора.

Как видаю из соотношения \eqref{eq:13}, пусковой ток клистрона тем меньше, чем меньше величина G. С ростом номера зоны самовозбуждение клистрона облегчается. Ток, требующийся для самовозбуждения клистрона, тем меньше, чем ниже ускоряющее напряжение $U_0$. Легче всего клистрон возбуждается в центрах зон. Напротив, на их краях 
$\sin \left( \theta _ { \text{з} } + \theta _ { \text{г} } \right) \rightarrow 0$ и 
$I _ { 0\, \text { пуск } } \rightarrow \infty$.

\section{Экспериментальная часть}

\begin{figure}[H]
	\centering
	\includegraphics[width=\textwidth]{text/fig4}
	\caption{ Блок-схема включения клистрона}
	\label{fig:4}
\end{figure}

Исследуемый клистрон предназначен для работы в десятисантиметровом диапазоне.

На рис. \ref{fig:4} показана схема включения клистрона. С блока питания (выпрямители I, II, III) напряжение подается на отражатель, резонатор и управляющий электрод. Контроль величин напряжений на электродах клистрона осуществляется по вольтметру, последовательно подключаемому с помощью
переключателя П1 к любому из электродов. К отражателю клистрона с по­мощью переключателя П2 может подключаться генератор пилообразного
напряжения; СВЧ колебания, генерируемые отражательным клистроном, с помощью петли связи через ответвитель подводятся к кристаллическому
детектору и волномеру ВМТ-10. В цепи детектора стоит микроамперметр, показания которого пропорциональны уровню выходной мощности клистрона
\footnote{При наблюдении зон генерации клистрона на экране осциллографа переключатель П2 ставится  в положение <<модуляция>> (<<Вкл>>), П3--в положение <<Выкл>>(<<Осциллограф>>)}.

\section{Эсперимент}
\subsection{Задания}
\begin{enumerate}
	\item  Включить установку и выставить рабочий режим клистрона, подо­
	брав подходящие значения напряжения на резонаторе (в интервале от 0 до
	-250 В), ускоряющем электроде (от 0 до + 150 В) и отражающем электроде
	(от 0 до —250 В).
	\item  Визуально исследовать режим генерации клистрона на экране осцил­лографа. Для этого подать на отражатель пилообразное напряжение, поста­
	вив переключатель П2 в положение «модуляция»: На вход Y осциллографа
	подается напряжение с детектора, а на вход X в качестве развертки — пи­лообразное напряжение с модулятора.
		\begin{enumerate}
			\item Используя кальку, зарисовать зоны генерации клистрона и пронумеро­
			вать их. Выяснить, как меняются зоны генерации в зависимости от потенци­
			алов электродов клистрона. Выбрать оптимальный режим работы клистро­на (найти значения напряжения на электродах, при которых реализуется наибольшая мощность в центре зон генерации и максимальная их ширина, т.е. максимальная частотная перестройка клистрона).
			\item Уменьшая амплитуду пилообразного напряжения на отражателе кли­
			клистрона, получить на экране только одну зону генерации. Определить с по­мощью волномера ширину частотной перестройки клистрона. Для этого, из­
			меняя настройку волномера, проследить на экране осциллографа движение
			метки по зоне генерации. Зафиксировав показания волномера в крайних
			точках зоны, определить ширину частотной перестройки клистрона вдоль
			зоны генерации.
		\end{enumerate}
	
	При выполнении остальных заданий переключатель П2 должен быть в
	положении «выкл», при этом должен быть выключен и осциллограф.

	\item Снять зависимости тока в цепи детектора (ток пропорционален мощ­ности колебаний) от: 
		\begin{enumerate}
			\item напряжения на отражателе (при нескольких фиксированных напряже­
			ниях на резонаторе),
			\item напряжения на резонаторе (при нескольких фиксированных напряже­
			ниях на отражателе).
		\end{enumerate} 
	Экспериментальные данные представить в виде графиков.
	
	Рекомендуется снятие характеристик в этом задании совмещать с изме­рением частотных зависимостей (см. задание 4).

	\item Измерить при помощи волномера длину волны колебаний, генерируе­
	мых клистроном, и проследить, как она меняется вдоль каждой зоны. Для
	этого снять зависимости частоты от:
		\begin{enumerate}
			\item  напряжения на отражателе (при фиксированном напряжении на резо­
			наторе) ,
			\item  напряжения на резонаторе (при фиксированном напряжении на отра­
			жателе).
		\end{enumerate}
	
	Экспериментальные данные представить в виде графиков.

	\item  Снять зависимость тока в цепи детектора от тока пучка для различных
	зон генерации клистрона. Потенциал отражателя для каждой зоны устанав­
	ливается по максимальной интенсивности колебаний (т.е. в центре зоны).
	Экспериментальные данные представить в виде графиков. На основании из­
	мерений определить пусковой ток клистрона для каждой зоны. Регулировку
	тока пучка осуществлять изменением величины потенциала на управляю­
	щем электроде.

	\item  Объяснить полученные экспериментальные данные.

\end{enumerate}

% \section{Заключение}
% \subsection{Контрольные вопросы}

% \begin{enumerate}
% 	\item Объясните принцип работы отражательного клистрона.
% 	\item Почему в приборах клистронного типа используются тороидальные ре­
% 	зонаторы?
% 	\item Можно ли для возвращения электронного потока в резонатор клистро­
% 	на использовать не постоянное электрическое, а постоянное магнитное поле?
% 	\item Что такое зоны генерации клистрона? Отличается ли структура поля
% 	в резонаторе для центров различных зон генерации?
% 	\item Как изменяется крутизна частотной перестройки клистрона с измене­
% 	нием номера зоны и при вариации добротности резонатора?
% \end{enumerate}
% \subsection{Литература}
% \begin{enumerate}
% 	\item  Гапонов В.И. Электроника. 4.11. М.: Физматгиз, 1960-
% 	\item  Лебедев И.В. Техника и приборы СВЧ. Ч.Н. М.: Высшая школа, 1972.
% 	\item  Коетиенко А.И. Введение в электронику СВЧ. М.: Изд. МГУ, 1989.
% 	\item  Градштейн И.С., Рыжик Н.М. Таблицы интегралов, сумм рядов и про­
% 	изведений. М., 1963. 5
% 	\item  Вайнштейн Л.А. Электромагнитные волны. М.: Радио и связь, 1988.
% \end{enumerate}
\subsection{Задание 2}
\begin{figure}[h]
	\centering
	\includegraphics[width=0.5\linewidth]{photo/img2}
	\caption{Характерный вид зон генерации клистрона на осциллографе}
	\label{fig:img1}
\end{figure}

Значения напряжения на электродах, при которых наблюдаются четыре зоны генерации клистрона:
\begin{itemize}
	\item $U_{\text{рез}}=162$ В
	\item $U_{\text{уск}}=120$ В
	\item $U_{\text{отр}}=90$ В
\end{itemize}

Увеличивая развертку на экране осциллографа, получили только одну зону генерации. Ширина частотной перестройки клистрона составила $\lambda=(10.64-10.46)\text{ см}=0.18$ см.
\subsection{Задание 3}
\begin{figure}[h]
		\centering
		\includegraphics[scale=1]{plots/task3a}
		\caption{Зависимость тока в цепи детектора от напряжения на отражателе}
		\label{fig:task3a}
\end{figure}
Из графиков видно, что чем ниже номер зоны генерации, тем шире сама зона. В условии возбуждения клистрона величина $\theta _ { \text{г} }$ зависит от напряжения отражателя:
\begin{equation*}
\theta _ { \text{г} } = \frac { 2 m } { e } \frac { v _ { 0 } \omega L } { U _ { 0 } - U _ { \text{ oтр } } }.
\end{equation*}
Здесь $U_{\text{отр}}<0$. Это значит, что чем больше по модулю $U_{\text{отр}}$, тем меньше $\theta _ { \text{г} }$ и больший диапазон $\theta _ { \text{г} }$ может убраться между $\pi+ 2\pi n$ и $2\pi(n+1)$.

\begin{figure}[h]
		\centering
		\includegraphics[scale=1]{plots/task3b}
		\caption{Зависимость тока в цепи детектора от напряжения на отражателе}
		\label{fig:task3b}
\end{figure}
В центре каждой зоны генерации поток отдает резонатору наибольшую мощность $\displaystyle P_{max}\sim I_1 U_m \sim \frac{2I_0 XJ_1(X) U_0}{\theta _ { \text{г} }}$, где $J_1(X)$ - функция Бесселя первого порядка. Первая гармоника конвекционного тока достигает максимума при $X=1.84$. Следовательно, активная мощность, отдаваемая электронным потоком резонатору, достигает максимума в любой зоне генерации при $X=1.84$. Это, в частности, означает, что колебательное напряжение на зазоре $U_m$, соответствующее максимуму электронной мощности, возрастает с уменьшением номера зоны. В соответствии с выражением для максимальной мощности, с уменьшением номера зоны возрастают также и абсолютные значения максимумов электронной мощности.
\begin{equation*}
U_m=\frac{2U_0X}{M\theta _ { \text{г} }}=\frac{1.84 U_o}{\pi M(n+\frac 34)n}
\end{equation*}

\subsection{Задание 4}

\begin{figure}[H]
		\centering
		\includegraphics[scale=1]{plots/task4a}
		\caption{Зависимость длины волны от напряжения на отражателе}
		\label{fig:task4a}
\end{figure}

При увеличении абсолютной величины $U_{\text{отр}}$ происходит рост частоты генерации клистрона. 

\begin{figure}[H]
		\centering
		\includegraphics[scale=1]{plots/task4b}
		\caption{Зависимость длины волны от напряжения на резонаторе}
		\label{fig:task4b}
\end{figure}

\subsection{Задание 5}
\begin{figure}[H]
		\centering
		\includegraphics[height=0.4\textheight]{plots/task5}
		\caption{Зависимость тока в цепи детектора от тока пучка для трех различных зон генерации}
		\label{fig:task5}
\end{figure}
\section{Заключение}
В ходе работы было изучено устройство и основные принципы действия отражательного клистрона. Для математического описания использовалась теория тороидального резонатора, входящего в состав клистрона. Рассматривался процесс динамического управления плотностью в электронном пучке, возбуждение переменным конвекционным током пучка колебаний в резонаторе. Рабочий режим соответствовал следующим напряжениям: 
\begin{itemize}
	\item $U_{\text{рез}}=162$ В
	\item $U_{\text{уск}}=120$ В
	\item $U_{\text{отр}}=90$ В
\end{itemize}

Получены зависимости тока в цепи детектора от напряжения на отражателе при постоянном напряжении на резонаторе и от напряжения на резонаторе при постоянном напряжении на отражателе. При помощи волномера измерена длина волны колебаний, создаваемых клистроном. А также, снята зависимость тока в цепи детектора от тока электронного пучка для трёх зон генератора. На основании экспериментальных данных были оценены ширина зазора в резонаторе и расстояние между резонатором и отражателем. Они получились равны соответственно  мм и  мм.
\end{document}